\documentclass[12pt,letterpaper]{article}
\title{\"Ubung 1}
\author{Shehata}
\date{}
\usepackage{mathtools}
\usepackage{graphicx}
\usepackage{longtable}
\usepackage{makecell, booktabs, caption}
\usepackage[german]{babel}
\usepackage{enumerate}
\usepackage{enumitem}
\usepackage{graphicx}
\usepackage{longtable}
\usepackage{titlesec}
\usepackage{makecell, booktabs, caption}
\usepackage{caption}
\usepackage{subcaption}
\usepackage{placeins}
\usepackage{subfiles}
\usepackage{import}
\usepackage{listings}
\usepackage[T1]{fontenc}
\usepackage{xcolor}

\definecolor{codegreen}{rgb}{0,0.6,0}
\definecolor{codegray}{rgb}{0.5,0.5,0.5}
\definecolor{codepurple}{rgb}{0.58,0,0.82}
\definecolor{backcolour}{rgb}{0.95,0.95,0.92}

\lstdefinestyle{mystyle}{
    backgroundcolor=\color{backcolour},   
    commentstyle=\color{codegreen},
    keywordstyle=\color{magenta},
    numberstyle=\tiny\color{codegray},
    stringstyle=\color{codepurple},
    basicstyle=\ttfamily\footnotesize,
    breakatwhitespace=false,         
    breaklines=true,                 
    captionpos=b,                    
    keepspaces=true,                 
    numbers=left,                    
    numbersep=5pt,                  
    showspaces=false,                
    showstringspaces=false,
    showtabs=false,                  
    tabsize=2
}

\renewcommand\theadfont{\normalsize\bfseries}
\renewcommand{\familydefault}{\sfdefault}
\vspace{6pt}

\begin{document}
\maketitle
\section{Entfaltung + Beweisen}

\subsection{Lösen Sie folgende Rekurrenz mittels Entfaltung:}

\begin{align*}
    T(1) & = 2                                            \\
    T(n) & = T(\frac{n}{3}) + 3 \quad \text{for n}\geq  2
\end{align*}

\paragraph{Resultat:}
\begin{align*}
    T(n) & = T(\frac{n}{3}) + 3                                         & i = 1 \\
    T(n) & = (T((\frac{\frac{n}{3}}{3}) + 3) + 3 = T(\frac{n}{9}) + 6   & i = 2 \\
    T(n) & = (T(\frac{\frac{n}{3}}{9}) + 6) + 3 = T(\frac{n}{27}) + 9   & i = 3 \\
    T(n) & = (T(\frac{\frac{n}{3}}{27}) + 9) + 3 = T(\frac{n}{81}) + 12 & i = 4 \\
         & T(\frac{n}{3^{i}}) + 3i
\end{align*}

\begin{align*}
    3^{i}    & = n                       & / \log     \\
    i\log(3) & = \log(n)                 & / :\log(3) \\
    i        & = \frac{\log(n)}{\log(3)} &            \\
    i        & = \log_3(n)               &
\end{align*}

\begin{align*}
     & 3\log_3(n) + T(\frac{n}{3^{\log_3(n)}})                                \\
     & 3\log_3(n) + T(\frac{n}{3^{\log_3(n)}}) & 3^{\log_3(n)} \Rightarrow  n \\
     & 3\log_3(n) + T(\frac{n}{n})             & T(1) \Rightarrow  2          \\
     & 3\log_3(n) + 2
\end{align*}


\subsection{Um welche Art von Algorithmus handelt es sich hier? Interpretieren Sie die Rekurrenz}

Die Rekursion drittelt mit jedem Aufruf die Datenmenge und f(n) ist immer 3.

\subsection{Beweisen Sie (mittels Raten und Beweisen), dass Ihr Ergebnis aus 1.a) richtig ist}
\paragraph{Assumption („wild guess“):}
$T(n) = 3 \log_3(n) + 2$
\paragraph{Proof by Induction:}\mbox{}\\
Base:
\begin{align*}
    T(1) = 3\log_3(1) + 2 = 3 \times 0 + 2 = 2
\end{align*}
Induction step:
\begin{align*}
    T(n) & = T(\frac{n}{3}) + 3                   \\
         & = (3 \log_3(\frac{n}{3}) + 2) + 3      \\
         & = (3 \log_3(\frac{n}{3}) + 2) + 3      \\
         & = (3 (\log_3(n) - \log_3(3))  + 2) + 3 \\
         & = (3 (\log_3(n) - 1)  + 2) + 3         \\
         & = (3\log_3(n) - 3  + 2) + 3            \\
         & = 3\log_3(n) - 1 + 3                   \\
         & = 3\log_3(n) + 2                       \\
\end{align*}

\pagebreak
\section{Master Theorem}
Lösen Sie folgende Rekurrenzen mit Hilfe des Master-Theorems:
\begin{enumerate}[label=(\alph*)]
    \item $T(n) = 4 T(\frac{n}{2}) + 2n^2 + 4n$
    \item $T(n) = 27 T(\frac{n}{3}) + n \log(n)$
    \item $T(n) = 8 T(\frac{n}{2}) + \frac{n^8}{2} - 2n$
\end{enumerate}
Geben Sie immer den Lösungsweg an und überprüfen Sie für den Fall, dass
$f(n) = \Omega(n^{log_b(a + \varepsilon)})$ für $\varepsilon > 0$ (Fall 3), auch die Zusatzbedingung.

\subsection*{a)}
$a = 4;\quad b = 2$ \\
$f(n) =  2n^2 + 4n$ \\
$\log_b(a) = \log_2(4) = 2$
\begin{enumerate}
    \item \begin{align*}
              f(n)      & = O(n^{\log_b(a) - \varepsilon})                               \\
              2n^2 + 4n & = O(n^{2 - \varepsilon})         & \rightarrow \text{Sum-Rule} \\
              2n^2      & \leq O(n^{2 - \varepsilon})                                    \\
              \text{Ist für } \varepsilon > 0 \text{ immer kleiner, also falsch}
          \end{align*}
    \item \begin{align*}
              k         & = 0                                                    \\
              f(n)      & = \Theta(n^{2\log^k(n)})                               \\
              2n^2 + 4n & = \Theta(n^2)            & \rightarrow \text{Sum-Rule} \\
              2n^2      & = \Theta(n^2)                                          \\
              \text{Es wächst gleich schnell, also korrekt}
          \end{align*}
\end{enumerate}
\paragraph{Ergebnis:}
$ \Theta(n \log_b(a)\log^{k+1}(n)) = \Theta(n^2 \log^1(n)) = \Theta(n^2 \log(n))$


\subsection*{b)}
$a = 27;\quad b = 3$ \\
$f(n) =  n \log(n)$
$\log_b(a) = \log_3(27) = 3$
\begin{enumerate}
    \item \begin{align*}
              f(n)      & = O(n^{\log_3(27) - \varepsilon}) \\
              n \log(n) & = O(n^{3 - \varepsilon})          \\
              n \log(n) & \leq O(n^{3 - \varepsilon})       \\
              \rightarrow \text{Es wächst schneller für } \varepsilon = 1
          \end{align*}
\end{enumerate}
\paragraph{Ergebnis:}
$ \Theta(n^3)$

\subsection*{c)}
$a = 8;\quad b = 2$ \\
$f(n) =  \frac{n^8}{2} - 2n$ \\
$\log_b(a) = \log_2(8) = 3$
\begin{enumerate}
    \item \begin{align*}
              f(n)               & = O(n^{\log_2(8) - \varepsilon})                               \\
              \frac{n^8}{2} - 2n & = O(n^{3 - \varepsilon})         & \rightarrow \text{Sum-Rule} \\
              \frac{n^8}{2}      & \leq O(n^{3 - \varepsilon})                                    \\
              \text{Ist für } \varepsilon > 0 \text{ immer kleiner, also falsch}
          \end{align*}
    \item \begin{align*}
              k                  & = 0                                                    \\
              f(n)               & = \Theta(n^{3\log^k(n)})                               \\
              \frac{n^8}{2} - 2n & = \Theta(n^3)            & \rightarrow \text{Sum-Rule} \\
              \frac{n^8}{2}      & = \Theta(n^3)                                          \\
              \text{Es wächst immer noch langsamer}
          \end{align*}
    \item \begin{align*}
              f(n)               & = \Omega(n^{\log_2(8) + \varepsilon})                               \\
              \frac{n^8}{2} - 2n & = \Omega(n^{3 + \varepsilon})         & \rightarrow \text{Sum-Rule} \\
              \frac{n^8}{2}      & \geq \Omega(n^{3 + \varepsilon})                                    \\
              \text{Die Aussage stimmt von } 0 < \varepsilon < 5
          \end{align*}
\end{enumerate}
\paragraph{Zusatzbedingung}
\begin{align}
    a f(\frac{n}{b})             & \leq c f(n)          \\
    8(\frac{\frac{n}{2} - 2n}{2})^{8} & \leq c \frac{n^8}{2} - 2n \\
    8(\frac{n}{4})^{8}           & \leq c \frac{n^8}{2} - 2n \\
    8\frac{n^8}{4^8}             & \leq c \frac{n^8}{2} - 2n \\
    8\frac{1}{4^8}               & \leq c \frac{1}{2} - 2n   \\
    \frac{8}{4^8}                & \leq \frac{c}{2} - 2n    \\
\end{align}
% n^{10}\frac{8}{2^{10}}       & \leq c               \\
% n^{10}\frac{8}{2^{10}}       &                      \\


\paragraph{Ergebnis:}
$ \Theta(n^9)$

\end{document}
